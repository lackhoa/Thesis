\section{Conclusion}
\ifthesis
This paper presented the following contributions:
\begin{enumerate}
\item A technique of using "reified if" to express failure in a relational manner.
\item The implementation a miniKanren variant to help extract certain "modes" of a program.
\item A way of reformatting staticKanren's answers to make it look more like the source program.
\end{enumerate}
The result of these contributions is the language staticKanren containing (besides mode extraction) a better presentation of answers and a more elegant core enabled by the omission of infinite streams. Additionally, staticKanren programs look a lot more like Scheme program. Perhaps this fact might be beneficial in making miniKanren programs more appealing to Scheme users.

The goal of this thesis is to explore techniques of avoiding code duplication of positive and negative relations. We learned the important lesson that while failure can be easily expressed with reified goal constructors such as \code{condo}, the behavior of the program is almost impossible to preserve. The best solution to this problem suggested by this paper is to specify the two-way algorithm, run it through staticKanren, and manually modify the generated result if needed.

This solution is not ideal, particularly due to the additional framework in staticKanren. However, I believe that it is a necessary price to pay for the mechanization of negation while retaining utmost user flexibility. An alternative approach no explored here is to use constructive negation (\cite{chan}), although it seems significantly more complex to implement.

There is an interesting use case of staticKanren in Scheme program analysis. With the help of \code{condo} it is very easy to convert any Scheme program into staticKanren and extract its various modes. This application was not explored in the paper due to time constraint, but it would be a nice topic for future work.
\else
\fi