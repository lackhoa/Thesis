\section{staticKanren}
This section introduces staticKanren, a language designed to analyze miniKanren programs. We will first step through some simple use cases in comparison to miniKanren so as to develop intuition for the more detailed implementation part. The last part demonstrates applications of the language.

\subsection{Introductory examples}
On the surface, staticKanren was designed to look as close to miniKanren as possible; however there are still some fundamental differences. Unlike miniKanren, we cannot model infinite answer streams with staticKanren. In fact, staticKanren programs can only return all answers using \code{run*} as there is no point keeping the other answers secrets. As a result, we cannot write recursive relations in staticKanren as is. On the bright side, this gives the language's implementation a conceptual clarity over miniKanren.

staticKanren offers only three primitives constraints: equality, disequality and \textbf{lazy} constraint. The first two retain their syntax and semantics from miniKanren, whereas the last is a new addition to emulate recursive relations in staticKanren. There is no reason to not include more constraints. The more knowledge staticKanren has, the better it is at its job. However, adding other constraints would be a distraction from our goal. We only demonstrate disequality in this paper as it plays such an important role in most miniKanren's programs.

This variant of the language also preserves all variable names created throughout the program. For example (from now until the end of the paper, the $\Rightarrow$ symbol is reserved for staticKanren programs unless stated otherwise):


\begin{lstlisting}
(run* (q p) (== q 4))
\end{lstlisting}
$\Rightarrow$ \code{(((4 #(p 0)) () ()))}

If the program above were run under miniKanren instead, the answer would have been \code{((4 _.0))}. As advertised, the variable \code{p} remains unknown so it gets to keep its name (let us ignore the zero on the right for now). Besides the query variables, staticKanren always returns two additional constraint stores, one for disequalities and one for fake constraints. Hence, every answer has the same shape of a three-element list.  The next example demonstrates how the language deals with name shadowing:

\begin{lstlisting}
(run* (q a) (fresh (a d) (== q `(,a . ,d))))
\end{lstlisting}
$\Rightarrow$ \code{((((#(a 1) . #(d 1)) #(a 0)) () ()))}

The same program under miniKanren would return \code{(((_.0 . _.1) _.2))} instead. There are two variables in the answer having the same name \code{a}. In the first answer, these two variables are still distinguished by their \textbf{birthdate} (i.e. the number on the right). Meanwhile, the second answer sidesteps this problem completely by introducing a new name for every unbound variables used in the answer. It is perhaps a subjective matter to debate whether the new representation is better, but there is one at least one theoretical problem avoided by the new approach\footnote{This example was even mentioned in one of the miniKanren's Uncourse Hangouts organized by William Byrd.}:

\begin{lstlisting}
(run* (q p) (== q '_.0))
\end{lstlisting}
$\Rightarrow$ \code{(((_.0 #(p 0)) () ()))}

The same program under miniKanren would return \code{((_.0 _.0))}!. And because two Scheme symbols are the same iff they look the same, there is no way to distinguish \code{p} and \code{q} in that case, even from the computer's perspective. This is not a fundamental flaw as there are still simple ways to distinguish reified names from Scheme symbols.

This last example demonstrates fake constraints. It is a silly program to stress that these fake constraints have no actual meaning.
\begin{lstlisting}
(run* (q p r) (fake `(mother ,q ,p)) (fake `(father ,p ,r))))
\end{lstlisting}

$\Rightarrow$
\begin{lstlisting}
(((#(q 0) #(p 0) #(r 0))
 ()
 ((father #(p 0) #(r 0)) (mother #(q 0) #(p 0)))))}
\end{lstlisting}

The \code{fake} form evaluate its body and add the resulting expression to the fake constraint store.

\subsection{Implementation}
This section presents an R6RS compliant implementation of staticKanren. We will highlight only those parts that demonstrates fundamental differences from the standard miniKanren implementation describe by \textcite{byrdphd}\footnote{However, this does not mean that merely overriding the definitions of miniKanren by the ones in this paper would give a working staticKanren since there are many other small differences (such as association representation).}. The full implementation of the code can be found at GITHUB URL GOES HERE!
Let us first define some macros (syntactic extensions) to make the code more compact. The pair of macro forms \code{letg@} and \code{lambdag@} helps extract various constraints from states. The "g" at the end is supposed to stand for "goal" (because goal constructors takes states as input), but it can be used in other cases.

The following functions construct and dispatch variables. Variables are vectors of two elements: its name (a symbol) and its birthdate (a number). Additionally, we would like to order variables. A variable is prioritized over another if it has a smaller birthdate (i.e. it was introduced sooner), or if it has the same birthdate and a lexicographically lesser name.
\begin{lstlisting}
(;; name is a symbol, bd is a number
 define var (lambda (name bd) (vector name bd)))
(define var->name (lambda (var) (vector-ref var 0)))
(define var->bd (lambda (var) (vector-ref var 1)))
(define var? vector?)
(define var<?
  ;; v1 is prioritized over v2
  (lambda (v1 v2)
    (let ([n1 (symbol->string (var->name v1))] [bd1 (var->bd v1)]
          [n2 (symbol->string (var->name v2))] [bd2 (var->bd v2)])
      (or (< bd1 bd2)
          (and (= bd1 bd2) (string<? n1 n2))))))
\end{lstlisting}

Next we have definitions about constraints. Many of the functions could easily be defined automatically using record, but we use list to retain conceptual simplicity and flexibility. A constraint is a 4-tuple consisting of a substitution \code{S}, a date counter \code{C}, a disequality store \code{D} and a fake constraint store \code{F}. The counter starts from 0 and counts upwards.
\begin{lstlisting}
(define all-constraints '(S C D F))
(define init-S '())
(define init-C 0)
(define init-D '())
(define init-F '())
(define make-c (lambda (S C D F) (list S C D F)))
(define init-c (make-c init-S init-C init-D init-F))
(define c->
  (lambda (c store)
    (rhs (assq store (map list all-constraints c)))))
(define update-S (lambda (c S) (letg@ (c : C D F) (make-c S C D F))))
(define update-C (lambda (c C) (letg@ (c : S D F) (make-c S C D F))))
(define update-D (lambda (c D) (letg@ (c : S C F) (make-c S C D F))))
(define update-F (lambda (c F) (letg@ (c : S C D) (make-c S C D F))))
\end{lstlisting}

The final piece of the groundwork concerns the answer stream monad. Because there are no lazy streams, they are much simpler than their counterparts in miniKanren. Even though they are trivial, we keep them here for the sake of comparison.
\begin{lstlisting}
(define mzero (lambda () '()))
(define unit (lambda (x) `(,x)))
(define choice (lambda (x y) `(,x . ,y)))
(define mplus append)
(define bind (λ (c* g) (apply mplus (map g c*))))
\end{lstlisting}

\code{unify} takes two terms along with a substitution and returns the extended substitution where these two terms are made the same. In the case that the two terms are always different, however, \code{unify} returns \code{#f}. There is a new clause in staticKanren to handle the case where both walked terms are variables, we make sure that seniors are always on the rhs. That way, walked seniors never result in juniors.
\begin{lstlisting}
(define unify
  (lambda (t1 t2 S)
    (let ([t1 (walk t1 S)]
          [t2 (walk t2 S)])
      (cond
       [(eq? t1 t2) S]
       [(and (var? t1) (var? t2))
        (or (and (var<? t2 t1)
                 (extend S t1 t2))
            (extend S t2 t1))]
       [(var? t1) (extend-check t1 t2 S)]
       [(var? t2) (extend-check t2 t1 S)]
       [(and (pair? t1) (pair? t2))
        (let ([S+ (unify (car t1) (car t2) S)])
          (and S+ (unify (cdr t1) (cdr t2) S+)))]
       [(equal? t1 t2) S]
       [else #f]))))
\end{lstlisting}

Now we are ready to define user-level functions, starting with fake constraint.
\begin{lstlisting}
(define fake
  (lambda (expr)
    (lambdag@ (c : F)
      (unit (update-F c `(,expr . ,F))))))
\end{lstlisting}

Next is \code{fresh}, a conjunction in the body of a \code{letv}. \code{letv} extracts the date counter from its input state and assign it to the new variables; the counter is then incremented in the resulting states.
\begin{lstlisting}
(define-syntax fresh
  (syntax-rules ()
    [(_ (x* ...) g g* ...)
     (letv (x* ...) (conj g g* ...))]))
(define-syntax letv
  (syntax-rules ()
    [(_ (x* ...) g)
     (lambdag@ (c : C)
       (let ([x* (var 'x* C)] ...)
         (g (update-C c (+ C 1)))))]))
\end{lstlisting}

The fascinating part is that reification can be made even simpler due to the fact that unknown variables get to keep their names. For the final result, \code{reify} only need to deep walk the query variables and the constraint stores in \code{S}. Additionally, a more compact representation of the disequality store is obtained by removing constraints that either contain no variable (i.e. it is grounded), or contain any irrelevant variable (i.e. variable that does not occur in the final version of \code{q*} or \code{F}).
\begin{lstlisting}
(define reify
  ;; This will return a c with clausal S
  (lambda (c q*)
    (letg@ (c : S D F)
      (let ([t (walk* q* S)]
            [D (walk* D S)]
            [F (walk* F S)])
        (let ([R (get-vars `(,t ,F))])
          (let ([D (rem-subsumed (purify-D D R))])
            `(,t ,D ,F)))))))
\end{lstlisting}

For closure, helpers of \code{reify} are given below. As an aside, we treat constraint stores as normal miniKanren terms; it explains why the code is so short.
\begin{lstlisting}
(define get-vars
  (lambda (t)
    (case-term t
      [v `(,v)]
      [(a d) (append (get-vars a) (get-vars d))]
      [a '()])))
(define purify-D
  (lambda (D R)
    (filter (lambda (d)
              (not (or (constant? d)
                       (has-iv? d R))))
            D)))
(define constant?
  (lambda (t)
    (case-term t
      [v #f]
      [(a d) (and (constant? a) (constant? d))]
      [atom #t])))
(define has-iv?
  (lambda (t R)
    (let has-iv? ([t t])
      (case-term t
        [v (not (memq v R))]
        [(a d) (or (has-iv? a) (has-iv? d))]
        [atom #f]))))
\end{lstlisting}

\subsection{Applications}